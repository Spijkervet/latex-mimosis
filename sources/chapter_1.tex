%%%%%%%%%%%%%%%%%%%%%%%%%%%%%%%%%%%%%%%%%%%%%%%%%%%%%%%%%%%%%%%%%%%%%%%%
\chapter{Introductie}
%%%%%%%%%%%%%%%%%%%%%%%%%%%%%%%%%%%%%%%%%%%%%%%%%%%%%%%%%%%%%%%%%%%%%%%%

\begin{center}
  \begin{minipage}{0.5\textwidth}
    \begin{small}
      ...
    \end{small}
  \end{minipage}
  \vspace{0.5cm}
\end{center}

%%%%%%%%%%%%%%%%%%%%%%%%%%%%%%%%%%%%%%%%%%%%%%%%%%%%%%%%%%%%%%%%%%%%%%%%
\section{Achtergrond}
%%%%%%%%%%%%%%%%%%%%%%%%%%%%%%%%%%%%%%%%%%%%%%%%%%%%%%%%%%%%%%%%%%%%%%%%


%%%%%%%%%%%%%%%%%%%%%%%%%%%%%%%%%%%%%%%%%%%%%%%%%%%%%%%%%%%%%%%%%%%%%%%%
\section{Onderzoeksgebied}
%%%%%%%%%%%%%%%%%%%%%%%%%%%%%%%%%%%%%%%%%%%%%%%%%%%%%%%%%%%%%%%%%%%%%%%%


%%%%%%%%%%%%%%%%%%%%%%%%%%%%%%%%%%%%%%%%%%%%%%%%%%%%%%%%%%%%%%%%%%%%%%%%
\section{Doel van het onderzoek}
%%%%%%%%%%%%%%%%%%%%%%%%%%%%%%%%%%%%%%%%%%%%%%%%%%%%%%%%%%%%%%%%%%%%%%%%

De opgave waar Prolody voor staat is meervoudig:
\begin{enumerate}[a)]
    \item het realiseren van een Nederlands sprekende, realistisch gesynthetiseerde stem met menselijk en persoonlijk karakter in de uitvoering van accent en dialect;
    \item uitvoering geven aan de ontwikkeling van deze stem op basis van hedendaagse technieken in Artificiële Intelligentie (AI);
    \item het samenstellen van, al dan niet aanwezige, kwantitatief en kwalitatief voldoende data om deze stem te realiseren;
    \item de aanpassing van prosodie op basis van de toestand waarin de conversatie verkeert, of wanneer een zin bedoeld wordt als vraag of opmerking.
\end{enumerate}

De haalbaarheidsstudie richt zich op onderdelen b, c en d en brengt kansen en belemmeringen in kaart voor het realiseren van onderdeel a.

Het eindrapport zal kunnen functioneren als onderlegger voor Prolody B.V. om een ontwikkelrichting te bepalen voor hun Robot Speech Project. Het rapport beoogt te kunnen beantwoorden wat de meest kansrijke vervolgstap is in het realiseren van een verbetering van het signaalpad tussen robot en mens, rekening houdend met:
\begin{itemize}
    \item de wensen vanuit Prolody B.V.;
    \item de vraag vanuit de markt;
    \item de kosten van de ontwikkelrichting
\end{itemize}

%%%%%%%%%%%%%%%%%%%%%%%%%%%%%%%%%%%%%%%%%%%%%%%%%%%%%%%%%%%%%%%%%%%%%%%%
\section{Geraadpleegde bronnen}
%%%%%%%%%%%%%%%%%%%%%%%%%%%%%%%%%%%%%%%%%%%%%%%%%%%%%%%%%%%%%%%%%%%%%%%%

Zie \ref{section:bibliography}

%%%%%%%%%%%%%%%%%%%%%%%%%%%%%%%%%%%%%%%%%%%%%%%%%%%%%%%%%%%%%%%%%%%%%%%%
\section{Opzet en leeswijzer}
%%%%%%%%%%%%%%%%%%%%%%%%%%%%%%%%%%%%%%%%%%%%%%%%%%%%%%%%%%%%%%%%%%%%%%%%
Het rapport is als volgt opgezet:
\paragraph{Hoofdstuk 2} besteedt aandacht aan de probleemstelling in text-to-speech systemen, de historie van de gesynthetiseerde stem en de hedendaags aanwezige technieken en initiatieven tot het realiseren van een menselijke stem met een eigen karakter. Het hoofdstuk wordt afgesloten met een afweging ten aanzien van de tegenwoordige initiatieven en de ontwikkelrichting van Prolody B.V.

\paragraph{Hoofdstuk 3} start met een kwantitatieve en kwalitatieve analyse van spraakdata. Deze analyse brengt de productiekaders in beeld zoals die thans worden gebruikt voor het genereren van een gesynthetiseerde stem. Er zal gefocust worden op de beschikbare Nederlandse spraakdata, waaronder een variëteit aan dialecten. Tevens zal gekeken worden naar de potentie om af te wijken van deze kaders op basis van financiële middelen en beschikbare spraakdata versus de kwaliteit van een gesynthetiseerde stem. Potentiële implementaties, technieken en modellen worden getoetst, op basis waarvan een kansrijk eindgebruik kan worden geformuleerd.

\paragraph{Hoofdstuk 4} begint met een toelichting op TODO gekozen ontwikkelvarianten die nadere beschouwing behoeven. Deze varianten worden elk geanalyseerd op hun technische en productionele haalbaarheid. Dit hoofdstuk vormt met deze technische verkenning de basis voor de financiële doorrekening in het volgende hoofdstuk.

\paragraph{Hoofdstuk 5} levert de financiële parameters waarmeer gerekend is en de voorwaarden die van toepassing zijn op de financiële doorrekening van de ontwikkelvarianten van het vorige hoofdstuk.

\paragraph{Hoofdstuk 6} trekt de conclusies op basis van de voorgaande hoofdstukken en geeft aanbevelingen voor het vervolg op deze haalbaarheidsstudie: de concrete uitwerking van de meest haalbaar geachte variant.