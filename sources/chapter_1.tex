%%%%%%%%%%%%%%%%%%%%%%%%%%%%%%%%%%%%%%%%%%%%%%%%%%%%%%%%%%%%%%%%%%%%%%%%
\chapter{Introductie}
%%%%%%%%%%%%%%%%%%%%%%%%%%%%%%%%%%%%%%%%%%%%%%%%%%%%%%%%%%%%%%%%%%%%%%%%

%%%%%%%%%%%%%%%%%%%%%%%%%%%%%%%%%%%%%%%%%%%%%%%%%%%%%%%%%%%%%%%%%%%%%%%%
\section{Achtergrond}
%%%%%%%%%%%%%%%%%%%%%%%%%%%%%%%%%%%%%%%%%%%%%%%%%%%%%%%%%%%%%%%%%%%%%%%%
Dit document onderzoekt de haalbaarheid van de ontwikkeling van een Nederlandse stem met verschillende accenten en dialecten op basis van de hedendaagse machine learning technieken. Dit verzoek is ontstaan vanuit de wens van Prolody B.V. een gesynthetiseerde stem te ontwikkelen voor een (zorg)robot dat het signaalpad tussen robot en mens verbeterd. Hedendaags zijn de beschikbare, gesynthetiseerde stemmen beperkt in hun gebruik. De stemmen zijn weinig tot niet expressief en bieden niet de mogelijkheid tot het aanpassen van prosodie of expressie door middel van contextuele factoren. Dit leidt tot communicatiefouten en een verstoring van het signaalpad. Omdat de sociale ontwikkeling tussen robots, machines of apparaten en mens onverminderd zal blijven groeien, is er een behoefte naar een oplossing voor dit probleem.

Voor het verbeteren van het signaalpad tussen robot en mens wordt daarom voorgesteld een Nederlands sprekende, gesynthetiseerde stem te ontwikkelen dat realistisch, natuurlijk en expressief in prosodie is en een uniek stemkarakter heeft.

%%%%%%%%%%%%%%%%%%%%%%%%%%%%%%%%%%%%%%%%%%%%%%%%%%%%%%%%%%%%%%%%%%%%%%%%
\section{Onderzoeksgebied}
%%%%%%%%%%%%%%%%%%%%%%%%%%%%%%%%%%%%%%%%%%%%%%%%%%%%%%%%%%%%%%%%%%%%%%%%
Dit onderzoek wordt uitgevoerd binnen het domein van spraaksynthese, en specifieker binnen het subdomein van modellen voor text-to-speech systemen.

%%%%%%%%%%%%%%%%%%%%%%%%%%%%%%%%%%%%%%%%%%%%%%%%%%%%%%%%%%%%%%%%%%%%%%%%
\section{Doel van het onderzoek}
%%%%%%%%%%%%%%%%%%%%%%%%%%%%%%%%%%%%%%%%%%%%%%%%%%%%%%%%%%%%%%%%%%%%%%%%

De opgave waar Prolody voor staat is meervoudig:
\begin{enumerate}[a)]
    \item het realiseren van een Nederlands sprekende, realistisch gesynthetiseerde stem met menselijk en persoonlijk karakter in de uitvoering van accent en dialect;
    \item uitvoering geven aan de ontwikkeling van deze stem op basis van hedendaagse technieken in Artificiële Intelligentie (AI);
    \item het samenstellen van, al dan niet aanwezige, kwantitatief en kwalitatief voldoende data om deze stem te realiseren;
    \item de aanpassing van prosodie op basis van de toestand waarin de conversatie verkeert, of wanneer een zin bedoeld wordt als vraag of opmerking.
\end{enumerate}

De haalbaarheidsstudie richt zich op onderdelen b, c en d en brengt kansen en belemmeringen in kaart voor het realiseren van onderdeel a.

Het eindrapport zal kunnen functioneren als onderlegger voor Prolody B.V. om een ontwikkelrichting te bepalen voor hun Robot Speech Project. Het rapport beoogt te kunnen beantwoorden wat de meest kansrijke vervolgstap is in het realiseren van een verbetering van het signaalpad tussen robot en mens, rekening houdend met:
\begin{itemize}
    \item de wensen vanuit Prolody B.V.;
    \item de vraag vanuit de markt;
    \item de kosten van de ontwikkelrichting
\end{itemize}

%%%%%%%%%%%%%%%%%%%%%%%%%%%%%%%%%%%%%%%%%%%%%%%%%%%%%%%%%%%%%%%%%%%%%%%%
\section{Geraadpleegde bronnen}
%%%%%%%%%%%%%%%%%%%%%%%%%%%%%%%%%%%%%%%%%%%%%%%%%%%%%%%%%%%%%%%%%%%%%%%%

Zie \ref{section:bibliography}

%%%%%%%%%%%%%%%%%%%%%%%%%%%%%%%%%%%%%%%%%%%%%%%%%%%%%%%%%%%%%%%%%%%%%%%%
\section{Opzet en leeswijzer}
%%%%%%%%%%%%%%%%%%%%%%%%%%%%%%%%%%%%%%%%%%%%%%%%%%%%%%%%%%%%%%%%%%%%%%%%
Het rapport is als volgt opgezet:
\paragraph{Hoofdstuk 2} besteedt aandacht aan de probleemstelling in text-to-speech systemen, de historie van de gesynthetiseerde stem en de hedendaags aanwezige technieken en initiatieven tot het realiseren van een menselijke stem met een eigen karakter. Het hoofdstuk wordt afgesloten met een afweging ten aanzien van de tegenwoordige initiatieven en de ontwikkelrichting van Prolody B.V.

\paragraph{Hoofdstuk 3} start met een kwantitatieve en kwalitatieve analyse van spraakdata. Deze analyse brengt de productiekaders in beeld zoals die thans worden gebruikt voor het genereren van een gesynthetiseerde stem. Er zal gefocust worden op de beschikbare Nederlandse spraakdata, waaronder een variëteit aan dialecten. Tevens zal gekeken worden naar de potentie om af te wijken van deze kaders op basis van financiële middelen en beschikbare spraakdata versus de kwaliteit van een gesynthetiseerde stem. Potentiële implementaties, technieken en modellen worden getoetst, op basis waarvan een kansrijk eindgebruik kan worden geformuleerd.

\paragraph{Hoofdstuk 4} begint met een toelichting op 4 gekozen ontwikkelvarianten die nadere beschouwing behoeven. Deze varianten worden elk geanalyseerd op hun technische en productionele haalbaarheid. Dit hoofdstuk vormt met deze technische verkenning de basis voor de financiële doorrekening in het volgende hoofdstuk.

\paragraph{Hoofdstuk 5} levert de financiële parameters waarmeer gerekend is en de voorwaarden die van toepassing zijn op de financiële doorrekening van de ontwikkelvarianten van het vorige hoofdstuk.

\paragraph{Hoofdstuk 6} trekt de conclusies op basis van de voorgaande hoofdstukken en geeft aanbevelingen voor het vervolg op deze haalbaarheidsstudie.