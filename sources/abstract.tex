\begin{center}
  \textsc{Abstract}
\end{center}
%
\noindent
%
Dit document is opgesteld om de haalbaarheid van de ontwikkeling van een Nederlandse stem met verschillende accenten en dialecten op basis van de hedendaagse machine learning technieken te onderzoeken. De volgende vragen staan centraal in dit onderzoek: Welke technieken liggen er voor handen? Welke mogelijkheden zijn er op gebied van flexibele aanpassing van de prosodie/uitspraak? Hoeveel uur spraakdata is er grofweg nodig en van welke kwaliteit moet die data ongeveer zijn? Op basis van de uitkomsten van deze vragen wordt een advies uitgebracht of binnen het project een prototype kan worden gemaakt, of dat er een andere focus nodig is voor het signaalpad tussen robot en mens.
